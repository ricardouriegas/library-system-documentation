\documentclass{scrreprt}
\usepackage{listings}
\usepackage{underscore}
\usepackage{graphicx}
\usepackage[bookmarks=true]{hyperref}
\usepackage[utf8]{inputenc}
\usepackage[english]{babel}
\hypersetup{
    bookmarks=false,    % show bookmarks bar?
    pdftitle={Software Requirement Specification},    % title
    pdfauthor={Jean-Philippe Eisenbarth},                     % author
    pdfsubject={TeX and LaTeX},                        % subject of the document
    pdfkeywords={TeX, LaTeX, graphics, images}, % list of keywords
    colorlinks=true,       % false: boxed links; true: colored links
    linkcolor=blue,       % color of internal links
    citecolor=black,       % color of links to bibliography
    filecolor=black,        % color of file links
    urlcolor=purple,        % color of external links
    linktoc=page            % only page is linked
}
\def\myversion{1.0 }
\date{}

\usepackage{hyperref}
\begin{document}

\begin{flushright}
    \rule{16cm}{5pt}\vskip1cm
    \begin{bfseries}
        \Huge{REQUERIMIENTOS DE\\ SOFTWARE}\\
        \vspace{1.5cm}
        para la\\
        \vspace{1.5cm}
        Biblioteca Publica del Partido Comunista\\
        \vspace{1.5cm}
        \LARGE{Version \myversion}\\
        \vspace{1.5cm}
        Escrito por: Uriegas Corporation\\
        % \vspace{1.5cm}
        % Submitted to : Fazle Mohammed Tawsif \\Lecturer\\
        \vspace{1.5cm}
        \today\\
    \end{bfseries}
\end{flushright}

\tableofcontents

\chapter{Introduction}

\section{Propósito}
Este documento tiene como objetivo describir los requisitos funcionales y no funcionales del \textbf{Sistema de Reservas de Biblioteca}. Los requisitos aquí documentados se han elaborado mediante entrevistas y sesiones de validación con los bibliotecarios, con el propósito de asegurar que el sistema cubra sus necesidades operativas.

\section{Alcance del sistema}
El sistema permitirá gestionar de manera eficiente las reservas y devoluciones de libros, facilitar el seguimiento del inventario, y enviar notificaciones automáticas. La solución mejorará la experiencia tanto de los usuarios como del personal bibliotecario.

\chapter{Requisitos Funcionales}
\begin{itemize}
    \item \textbf{RF1. Registro de Usuarios}
    \begin{itemize}
        \item \textbf{Descripción:} El sistema permitirá que los usuarios se registren proporcionando su nombre, correo electrónico y una contraseña segura.
        \item \textbf{Validación:} El sistema verificará que el correo no esté registrado previamente y que la contraseña cumpla con los criterios de seguridad.
    \end{itemize}
    \item \textbf{RF2. Inicio de Sesión}
    \begin{itemize}
        \item \textbf{Descripción:} El sistema permitirá que los usuarios registrados inicien sesión usando su correo y contraseña.
        \item \textbf{Regla:} Si se ingresan credenciales incorrectas tres veces, la cuenta se bloqueará temporalmente.
    \end{itemize}
    \item \textbf{RF3. Búsqueda de Libros}
    \begin{itemize}
        \item \textbf{Descripción:} Los usuarios podrán buscar libros por título, autor o categoría.
        \item \textbf{Regla:} Si no se encuentran libros que coincidan con los criterios, se mostrará un mensaje informativo.
    \end{itemize}
    \item \textbf{RF4. Reserva de Libros}
    \begin{itemize}
        \item \textbf{Descripción:} Los usuarios registrados podrán reservar libros disponibles y seleccionar una fecha para su recogida.
        \item \textbf{Reglas:}
        \begin{itemize}
            \item Si el libro ya está reservado, no se permitirá otra reserva hasta que se devuelva.
            \item El sistema establecerá un \textit{tiempo límite de 5 días} para recoger la reserva, después del cual el libro quedará disponible nuevamente.
        \end{itemize}
    \end{itemize}
    \item \textbf{RF5. Gestión de Devoluciones}
    \begin{itemize}
        \item \textbf{Descripción:} El bibliotecario podrá registrar la devolución de libros mediante escaneo del código de barras o introducción manual.
        \item \textbf{Regla:} El sistema actualizará la disponibilidad del libro automáticamente.
    \end{itemize}
    \item \textbf{RF6. Envío de Notificaciones}
    \begin{itemize}
        \item \textbf{Descripción:} El sistema enviará notificaciones automáticas cuando:
        \begin{itemize}
            \item Un libro reservado esté listo para ser recogido.
            \item La reserva esté próxima a vencer.
        \end{itemize}
    \end{itemize}
    \item \textbf{RF7. Historial de Reservas}
    \begin{itemize}
        \item \textbf{Descripción:} Los usuarios podrán consultar su historial de reservas y ver el estado (pendiente, recogido, devuelto).
        \item \textbf{Regla:} Se permitirá filtrar las reservas por estado y fecha.
    \end{itemize}
    \item \textbf{RF8. Administración del Inventario}
    \begin{itemize}
        \item \textbf{Descripción:} Los bibliotecarios podrán añadir, modificar o eliminar libros del catálogo.
        \item \textbf{Regla:} Si los datos introducidos son incompletos, el sistema mostrará un mensaje de error.
    \end{itemize}
\end{itemize}

\chapter{Requisitos No Funcionales}
\begin{itemize}
    \item \textbf{RNF1. Seguridad - Autenticación Segura}
    \begin{itemize}
        \item \textbf{Descripción:} El sistema HTTPS y cifrado de contraseñas para proteger las credenciales de los usuarios.
        \item \textbf{Regla:} Si se intenta acceder desde una conexión no segura, se bloqueará la operación.
    \end{itemize}
    \item \textbf{RNF2. Usabilidad - Interfaz Intuitiva}
    \begin{itemize}
        \item \textbf{Descripción:} El proceso de búsqueda y reserva de libros debe completarse en \textit{menos de tres pasos}.
        \item \textbf{Regla:} El diseño debe ser intuitivo y accesible desde dispositivos móviles y navegadores.
    \end{itemize}
    \item \textbf{RNF3. Compatibilidad - Navegadores y Dispositivos}
    \begin{itemize}
        \item \textbf{Descripción:} El sistema debe ser compatible con \textit{Chrome, Firefox, Safari} y funcionar correctamente en dispositivos móviles.
    \end{itemize}
    \item \textbf{RNF4. Escalabilidad - Manejo de Usuarios Simultáneos}
    \begin{itemize}
        \item \textbf{Descripción:} El sistema debe soportar hasta \textit{10,000} usuarios simultáneos sin afectación del rendimiento.
    \end{itemize}
    \item \textbf{RNF5. Rendimiento - Respuesta Rápida}
    \begin{itemize}
        \item \textbf{Descripción:} Las consultas de búsqueda deben devolver resultados en \textit{menos de 2 segundos}.
    \end{itemize}
    \item \textbf{RNF6. Alta Disponibilidad}
    \begin{itemize}
        \item \textbf{Descripción:} El sistema debe estar disponible el \textit{99.9\% del tiempo, con un máximo de 1 hora de inactividad} mensual.
    \end{itemize}
    \item \textbf{RNF7. Cumplimiento Legal - Protección de Datos}
    \begin{itemize}
        \item \textbf{Descripción:} El sistema debe cumplir con normativas de protección de datos personales, permitiendo que los usuarios soliciten la eliminación de su información en cualquier momento.
    \end{itemize}
\end{itemize}

\chapter{Casos de Uso}
\section{Casos de uso funcionales}

\subsection{Caso de uso 1: Registro de usuarios}
\textbf{Actor}: Usuario no registrado \\
\textbf{Precondición}: El usuario no debe tener una cuenta registrada en el sistema \\
\textbf{Descripción}: Permitir que un usuario se registre en el sistema proporcionando su nombre, correo electrónico y una contraseña segura.

\subsubsection{Flujo principal}
\begin{enumerate}
    \item El usuario accede al formulario de registro.
    \item El usuario introduce su nombre, correo electrónico y contraseña.
    \item El sistema valida que el correo no esté registrado previamente.
    \item El sistema valida la complejidad de la contraseña.
    \item El sistema almacena los datos y envía un correo de verificación.
    \item El usuario recibe el correo y confirma su cuenta haciendo clic en el enlace de activación.
\end{enumerate}

\subsubsection{Excepciones}
\begin{itemize}
    \item 3a. El correo ya está registrado: El sistema muestra un mensaje de error.
    \item 4a. La contraseña no cumple con los criterios: El sistema solicita una contraseña más segura.
\end{itemize}

\textbf{Postcondición}: El usuario se registra con éxito y puede iniciar sesión después de verificar su cuenta por correo.

\subsection{Caso de uso 2: Iniciar sesión}
\textbf{Actor}: Usuario registrado \\
\textbf{Precondición}: El usuario debe haberse registrado previamente y haber confirmado su cuenta. \\
\textbf{Descripción}: Permitir que un usuario registrado inicie sesión en el sistema usando su correo electrónico y contraseña.

\subsubsection{Flujo principal}
\begin{enumerate}
    \item El usuario accede a la página de inicio de sesión.
    \item Introduce su correo electrónico y contraseña.
    \item El sistema valida las credenciales.
    \item El sistema otorga acceso a la cuenta del usuario.
\end{enumerate}

\subsubsection{Excepciones}
\begin{itemize}
    \item 3a. Credenciales incorrectas: El sistema muestra un mensaje de error.
    \item 3b. El usuario intenta iniciar sesión múltiples veces con credenciales incorrectas: El sistema bloquea temporalmente la cuenta.
\end{itemize}

\textbf{Postcondición}: El usuario inicia sesión exitosamente y accede a su cuenta.

\subsection{Caso de uso 3: Búsqueda de libros}
\textbf{Actor}: Usuario registrado \\
\textbf{Precondición}: El catálogo de libros debe estar disponible en el sistema. \\
\textbf{Descripción}: Permitir que los usuarios busquen libros por título, autor o categoría en el catálogo de la biblioteca.

\subsubsection{Flujo principal}
\begin{enumerate}
    \item El usuario accede al catálogo de libros.
    \item El usuario introduce el criterio de búsqueda (título, autor, categoría).
    \item El sistema muestra una lista de libros que coinciden con el criterio de búsqueda.
    \item El usuario selecciona un libro para ver más detalles.
\end{enumerate}

\subsubsection{Excepciones}
\begin{itemize}
    \item 3a. No se encuentran libros que coincidan con el criterio: El sistema muestra un mensaje indicando que no hay resultados.
\end{itemize}

\textbf{Postcondición}: El usuario visualiza los resultados de la búsqueda y puede acceder a la información del libro.

\subsection{Caso de uso 4: Reserva de libros}
\textbf{Actor}: Usuario registrado \\
\textbf{Precondición}: El usuario debe haber iniciado sesión en el sistema. El libro debe estar disponible para reserva. \\
\textbf{Descripción}: Permitir que los usuarios reserven libros disponibles para recogerlos en la biblioteca en una fecha determinada.

\subsubsection{Flujo principal}
\begin{enumerate}
    \item El usuario busca un libro y lo selecciona.
    \item El sistema muestra si el libro está disponible para reservar.
    \item El usuario selecciona la opción de "Reserva".
    \item El sistema solicita al usuario elegir una fecha de recogida.
    \item El sistema confirma la reserva del libro y muestra la fecha de recogida.
\end{enumerate}

\subsubsection{Excepciones}
\begin{itemize}
    \item 2a. El libro no está disponible: El sistema muestra un mensaje indicando que el libro no está disponible para reservar.
    \item 5a. El usuario no selecciona una fecha de recogida: El sistema no permite proceder con la reserva.
\end{itemize}

\textbf{Postcondición}: El usuario ha reservado el libro exitosamente y puede recogerlo en la fecha seleccionada.

\subsection{Caso de uso 5: Notificaciones}
\textbf{Actor}: Sistema, Usuario registrado \\
\textbf{Precondición}: El usuario debe haber realizado una reserva o estar en una lista de espera. \\
\textbf{Descripción}: El sistema debe enviar notificaciones automáticas por correo electrónico cuando una reserva esté lista para recogerse o cuando un libro reservado esté disponible.

\subsubsection{Flujo principal}
\begin{enumerate}
    \item El usuario reserva un libro.
    \item El sistema monitorea el estado de la reserva.
    \item Cuando el libro está disponible para recogida, el sistema envía una notificación por correo electrónico al usuario.
    \item El usuario recibe el correo y procede a la biblioteca a recoger el libro.
\end{enumerate}

\subsubsection{Excepciones}
\begin{itemize}
    \item 4a. El correo no se entrega correctamente: El usuario no recibe la notificación, pero puede consultar el estado de su reserva en el sistema.
\end{itemize}

\textbf{Postcondición}: El usuario recibe una notificación por correo electrónico cuando su reserva está lista.

\subsection{Caso de uso 6: Historial de reservas}
\textbf{Actor}: Usuario registrado \\
\textbf{Precondición}: El usuario debe haber iniciado sesión en el sistema y tener un historial de reservas. \\
\textbf{Descripción}: Permitir que los usuarios consulten su historial de reservas, incluyendo el estado de cada una (pendiente, recogido, devuelto).

\subsubsection{Flujo principal}
\begin{enumerate}
    \item El usuario accede a su cuenta.
    \item El usuario selecciona la opción "historial de reservas".
    \item El sistema muestra una lista de todas las reservas realizadas por el usuario, junto con su estado.
    \item El usuario puede filtrar las reservas por estado o fecha.
\end{enumerate}

\subsubsection{Excepciones}
\begin{itemize}
    \item 3a. El usuario no tiene reservas en su historial: El sistema muestra un mensaje indicando que no hay reservas.
\end{itemize}

\textbf{Postcondición}: El usuario consulta el historial de reservas.

\subsection{Caso de uso 7: Administración de inventarios}
\textbf{Actor}: Bibliotecario \\
\textbf{Precondición}: El bibliotecario debe haber iniciado sesión en el sistema con permisos de administración. \\
\textbf{Descripción}: Permitir a los bibliotecarios añadir, eliminar o modificar los registros de libros disponibles en el catálogo de la biblioteca.

\subsubsection{Flujo principal}
\begin{enumerate}
    \item El bibliotecario accede a la sección de administración de inventarios.
    \item El bibliotecario selecciona la opción de "añadir", "eliminar" o "modificar" un libro.
    \item El sistema solicita los detalles del libro.
    \item El bibliotecario introduce o actualiza la información requerida.
    \item El sistema confirma la acción y actualiza el catálogo de libros.
\end{enumerate}

\subsubsection{Excepciones}
\begin{itemize}
    \item 3a. El bibliotecario introduce datos incorrectos o incompletos: El sistema muestra un mensaje de error y no permite continuar.
\end{itemize}

\textbf{Postcondición}: El bibliotecario gestiona el catálogo de libros de la biblioteca.

\subsection{Caso de uso 8: Devoluciones de los libros}
\textbf{Actor}: Usuario registrado, bibliotecario \\
\textbf{Precondición}: El usuario debe haber tomado prestado un libro y estar en proceso de devolución. \\
\textbf{Descripción}: Registrar la devolución de libros por parte de los usuarios y actualizar la disponibilidad de los libros en el catálogo.

\subsubsection{Flujo principal}
\begin{enumerate}
    \item El usuario acude a la biblioteca para devolver un libro.
    \item El bibliotecario escanea el código de barras del libro.
    \item El sistema actualiza el estado del libro a "disponible" en el catálogo.
    \item El sistema notifica al usuario que la devolución ha sido exitosa.
\end{enumerate}

\subsubsection{Excepciones}
\begin{itemize}
    \item 2a. El sistema no puede leer el código de barras: El bibliotecario ingresa manualmente la información.
\end{itemize}

\textbf{Postcondición}: El libro se devuelve correctamente y se actualiza la disponibilidad en el catálogo.

\section{Casos de uso no funcionales}

\subsection{Caso de uso no funcional 1: Seguridad}
\textbf{Descripción}: Asegurar que el sistema cuente con medidas de seguridad, como encriptación de contraseñas y autenticación de dos factores.

\subsection{Caso de uso no funcional 2: Rendimiento}
\textbf{Descripción}: Garantizar que el sistema pueda manejar una carga alta de usuarios sin deterioro en el rendimiento.

\subsection{Caso de uso no funcional 3: Disponibilidad}
\textbf{Descripción}: El sistema debe estar disponible para los usuarios las 24 horas del día, los 7 días de la semana.

\subsection{Caso de uso no funcional 4: Usabilidad}
\textbf{Descripción}: Asegurar que la interfaz del sistema sea fácil de usar, intuitiva y accesible para todos los usuarios.

\subsection{Caso de uso no funcional 5: Compatibilidad}
\textbf{Descripción}: El sistema debe ser compatible con diferentes dispositivos y navegadores web.



\chapter{Riesgos Identificados}
\begin{itemize}
    \item \textbf{R1:} El usuario no recoge el libro reservado a tiempo.
    \begin{itemize}
        \item \textbf{Mitigación:} Se establecerá un tiempo límite para las reservas.
    \end{itemize}
    \item \textbf{R2:} El sistema no lee correctamente el código de barras.
    \begin{itemize}
        \item \textbf{Mitigación:} Se habilitará la opción de registro manual de devoluciones.
    \end{itemize}
    \item \textbf{R3:} Sobrecarga del servidor en horas pico.
    \begin{itemize}
        \item \textbf{Mitigación:} Se implementarán servidores de respaldo para distribuir la carga.
    \end{itemize}
\end{itemize}

\chapter{Aprobación del Documento}
Este documento ha sido revisado y validado con los bibliotecarios. Con la aprobación de está versión, se procederá al desarrollo del sistema de reservas.

\begin{tabular}{|l|l|l|l|}
    \hline
    \textbf{Nombre} & \textbf{Rol} & \textbf{Firma} & \textbf{Fecha} \\ \hline
    Bibliotecario 1 & Responsable de reservas & & \\ \hline
    Bibliotecario 2 & Encargado de Inventario & & \\ \hline
    Analista de Sistemas & Encargado del Proyecto & & \\ \hline
\end{tabular}
    

\chapter{Conclusión}
Este documento refleja los requisitos funcionales y no funcionales para el sistema de reservas de biblioteca. La implementación de este sistema permitirá optimizar la gestión de reservas, devoluciones e inventario, mejorando la experiencia del usuario y del personal de la biblioteca.

\chapter{Anexos}
\section{Prototipo de la interfaz} 
% anexar el prototipo de la interfaz aqui

\section{Diagrama de Casos de Uso} 
% anexar el diagrama de casos de uso aqui

\section{Manual preliminar del Usuario} 
% anexar el manual del usuario aqui

\end{document}